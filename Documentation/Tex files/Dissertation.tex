\documentclass[lettersize,journal]{IEEEtran}
\usepackage{amsmath,amsfonts}
\usepackage{algorithmic}
\usepackage{algorithm}
\usepackage{array}
\usepackage[caption=false,font=normalsize,labelfont=sf,textfont=sf]{subfig}
\usepackage{textcomp}
\usepackage{stfloats}
\usepackage{url}
\usepackage{verbatim}
\usepackage{graphicx}
\usepackage{cite}
\usepackage[utf8]{inputenc}
\usepackage{listings}
\usepackage{array}
\usepackage{makecell}

\renewcommand\theadalign{bc}
\renewcommand\theadfont{\bfseries}
\renewcommand\theadgape{\Gape[4pt]}
\renewcommand\cellgape{\Gape[4pt]}
\hyphenation{op-tical net-works semi-conduc-tor IEEE-Xplore}
%\usepackage{tikz,graphics,color,fullpage,float,epsf,caption}
% updated with editorial comments 8/9/2021

\begin{document}

\title{The Risks and Benefits of Social Media, and Its Place in Higher Education}

\author{DD252935}
% <-this % stops a space


%\thanks{This paper was produced by the IEEE Publication Technology Group. They are in Piscataway, NJ.}
% <-this % stops a space
%\thanks{Manuscript received April 19, 2021; revised August 16, 2021.}

% The paper headers
%\markboth{Journal of \LaTeX\ Class Files,~Vol.~14, No.~8, August~2021}
%{Shell \MakeLowercase{\textit{et al.}}: A Sample Article Using IEEEtran.cls for IEEE Journal
%\IEEEpubid{0000--0000/00\$00.00~\copyright~2021 IEEE}

% Remember, if you use this you must call \IEEEpubidadjcol in the second
% column for its text to clear the IEEEpubid mark.
\maketitle

\begin{abstract}
	This study examines the effectiveness of a purpose-built social media platform for supporting learning and social interaction in higher education. A mixed-methods
	approach was used, including qualitative and quantitative surveys, to compare the new platform to the student portal. The results indicate that the social media
	platfrom showed promise for enhancing student engagement and collaboration, but further research is needed to fully evaluate its potential. These findings suggest 
	that social media-style platforms could be a valuable addition to traditional student content delivery systems.


        \end{abstract}

\begin{IEEEkeywords}
Social Media, Social Networking, Higher Education.
\end{IEEEkeywords}

\section{Introduction}


    This paper looks at the utility that social media platforms can provide in a university setting as a means to justify, design and develop an institutional social media site to aid in the 
    academic and social aspects of university life.


    Social media is all around us, and the vast majority of us use social media in some way or another very
    frequently. Many studies have taken place to explore the impact social media could have on students
    when they have been encouraged to use existing platforms as a contact and collaboration tool as part
    of their course.

    Finding your place socially at university can be very daunting, especially if you have been unable to find
    your way into any large social events, or onto any student-run social channels such as Discord \cite{Discord} etc, if any
    such things are in place at all. Failure to find such places can have a major impact on not only the university
    experience but also their mental health, as they can find themselves isolated.

    Lockdowns and isolations during the pandemic left many students segregated and created an environment in which
    many had left home to find themselves unable to socialise and find their new student cohort, as such, there is a need for better online socialization tools to be adopted by universities.

    This paper researches into the question; could and institutionalised social media platform benefit the academic and social aspects of university life?

 \subsection{Roadmap of this paper}
    In section II a review of past work and literature is conducted to first understand how social media has been implemented in a university setting, followed by an observation of 
    previous literature exploring the effects of social in an effort to comprehend the impacts it can have to ultimately develop an artefact that is positive in nature.

    Section III is an evaluation of both tiers of the related work.
    Section IV outlines the design elements of the prototype social media platform artefact, including user interface elements 
    and technological stack. Section V looks at the development process of the social media platform as well as the testing of artefact performance.

    Section VI explains the research methodology chosen for this 
    study including legal, social and ethical considerations. Section VII observes the experimental results. Section VIII discusses the results and implications of this paper and contained research. 
	Section IX looks at the potential for future study within this area accounting for limitations encountered within this iteration and Section X concludes with a summary of findings.



\section{Related Work}
    This literature review investigates the risks and benefits attached to
    social media and the potential advantages that it could bring forward as a
    tool in higher education and pedagogy. Social media has made a massive
    impact on society in many ways, and using it one way or another has become
    commonplace in most of our lives, but do we fully understand the risks and
    advantages that it presents? This literary analysis of recent (2010-2022)
    research papers aim to explore the findings on the possible side effects of
    social media to weigh the pros against the cons regarding
    the integration of social media with higher education (HE) and pedagogy. I
    hypothesize, that with proper application, social media could become a valuable
    tool within HE institutions and could help increase engagement with learning
    materials and courses.

\subsection{Social Media in Higher Education}
    Liu \cite{Liu2010} acknowledges that each social media platform comes with
    its own set of strengths and weaknesses and that the integration of such into
    pedagogy must be planned cautiously, ensuring that it is the strengths of the platform
    that are leveraged and not the potential distractions and difficulties that could
    hinder student learning. Liu talks of each social media platform being a tool,
    each in its own specific right and each with its designated purpose, so a one size
    fits all approach would only bring about nuisance. The author notes, for instance,
    that we could capitalize on Facebook's ubiquity and capabilities for collaboration.
    Liu \cite{Liu2010} and Baruah \cite{Baruah2012} both talk about the integration of
    social media into higher education and both conclude by sharing their thoughts on that
    it would be an advantage to implement social media elements as tools within higher
    education. Baruah further empowers Liu's point about different platforms providing
    different tools, by discussing how much easier collaboration becomes when using
    online facilities. Online mediums that provide features allowing users to co-draft
    documents, organise members, arrange meetings, spread information, and gauge opinion,
    all while having the capability to reach audiences all over the world.
    Baruah concludes that there will be a greater capacity for groups to participate in
    collective action, going on to say that it is the hallmark of civil society.

    Kelm \cite{Kelm2011} also implemented social media into their course and noticed an
    increase in engagement from their students and reported a greater sense of team ethic
    between classmates. Kelm concluded with a note stating that the secret for educators is
    to observe how technology is used in everyday life and then implement that use into
    our education systems.
    Wang et al. \cite{Wang2011} mention in their paper that there is a call for an approach
    to try and better balance the relationship between social media and academic study but
    pays a great deal of respect to the potential benefits that it can offer. The paper goes
    on to mention that students are very likely to be affected by social media, whilst it
    provides a world in which to make new friends and release pressure, it can absolutely
    impact students' lives and grades, calling for the aforementioned balance.

    Evans \cite{Evans2014} encouraged students to interact with him and their peers through
    Twitter and found that the amount of Twitter usage was associated with increased student
    engagement. Course-related tweeting showed no evidence of being related to interpersonal
    relations between students and their tutors, and finally that Twitter usage did not relate
    to class attendance.

    Williams \cite{Williams2022} talks of the capabilities that social media brings
    forward as advantages in enhancing learner engagement in a very efficient way
    and reiterates the points provided by Junco et al \cite{Junco et al 2013}.
    The paper from Junco et al. follows a similar experiment to Evans and his 2014
    paper \cite{Evans2014} but in a slightly more robust and comprehensive fashion.
    This was achieved by using two separate groups, the first consisting of 125
    students, half of whom were required to use Twitter while the other half were
    required to use Ning, whereas the participation of Twitter and Ning usage was
    voluntary for study group 2. The study recognised greater motivation towards
    engagement from study group 1 (those required to use Twitter and Ning). The
    paper concludes by stating that new technologies being incorporated into contemporary
    classrooms is an important development in an effort to produce more effective
    learning strategies and outcomes, while calling for contemporary students to
    improve their capacity to engage in more self-directed collaborative practices
    in order to better take ownership of their learning.

    Tripathi \cite{Tripathi 2022} observed that nearly two-thirds of the faculty at
    their institution had used social media in a class session, some even
    posting content for students to further read outside of classes, which saw
    promising levels of engagement while other members of the faculty ask students
    explicitly to utilise social media as part of course assignments. On an end
    note the paper reaffirms that the presence of social media within HE is
    increasingly visible as instructors continue to further employ technology
    to enhance their teaching methods and promote active learning for students.

    Haythornthwaite, Paulin, and Gruzd \cite{Haythornthwaite et al 2016} discuss
    an overview of the measures and potential of a multi-method approach for studying learning
    through means of social media, based on a workshop held at the 2014 Learning
    Analytics and Knowledge conference. The paper pays vast respect to the
    implementation of social media into both teaching and learning being new,
    but still advancing rapidly. It is recognised that learners are already
    present on these channels and are already capable of information search and
    acquisition, learning community support, knowledge building, and engagement.
    In one of the final notes of the paper, there is mention that different
    settings of formality would call for different considerations to be made. In
    a formal setting, the intent of the instructor must be taken into
    consideration while examining the discussion formation comparatively against
    the desired communication and pedagogical outcomes. Whereas in more informal
    settings, we must consider the impact of things on a more societal level of
    mass learning and how the balance of the development of sustained learning
    communities is affected by massively distributed learning and the
    'just-in-time learning' associated with social media exchanges.

\subsection{The Effects of Social Media}
    The paper by Amedie \cite{Amedie 2015} mentions that, ironically, social
    media is in effect turning us into one of the most antisocial generations
    yet. The paper talks about the connection between social media and anxiety
    – It states that social media causes depression and anxiety in two ways. Chronic
    stress causes depression and anxiety. Being constantly alert for new social
    media messages, to your instinctive fight-or-flight limbic system, is the 
    same as being on continuous alert for predators, which causes a release of
    the stress hormone cortisol. The second cause of depression anxiety is
    constantly trying to maintain an unrealistic and unachievable image of
    oneself on their chosen social network. The paper also mentions
    that social media can pave the way for criminal activity, by putting to use
    the freedoms offered by social media to hide their identity and engage in
    things like cyberbullying, cyber terrorism, human trafficking and drug dealing,
    though only talks in depth of cyberbullying, criminal and terrorist activities
    as they are the most common illicit activities. Amedie concludes that despite
    the positive benefit of rapid information sharing, social media enables people
    to create false identities and superficial connections, causes depression and
    is a primary recruiting tool for criminals and terrorists. It also mentions
    that the negative impacts of social media are rarely discussed, while the benefits
    are often emphasized.

    Kuppuswamy and Narayan \cite{Kuppuswamy et al 2010} recognise that social
    media sites provide function for individuals to create and maintain social
    ties, which can be of great benefit in both academic and social settings.
    It is also observed that these same sites present a risk to individuals'
    privacy, health, safety and professional reputations if the platforms are
    not used responsibly.

    In a 2012 paper by Tariq et al \cite{Tariq et al 2012}, the authors observed that more than 90\%
    of college students use social media \cite{Ellison et al 2007} and they found social media to be
    having a negative impact on education. Tariq et al believe this to be due to social networks
    capturing the total attention of their users and redirecting them towards non-educational, inappropriate
    and unethical activities such as ``useless chatting, time killing by random searching and not doing their jobs". The paper
    goes on to note that social networking sites quite often play host to attractive activities such
    as gaming or advertisements, enticing people to sign up or simply waste time, it is the over-indulgence
    of such activities that cause users to develop social media addiction. It states that providing the ubiquitous
    facilities of social networks is a straight invitation to addiction for any teenager and even an adult, as
    academic satisfaction is not enough got those students who suffer from social isolation\cite{Pempek et al 2009}. 

    A study conducted by W.Akram and R.Kumar \cite{Akram et al 2017} observes both the negative
    and positive impacts of social media on society and business.  The paper notes the merits presented by social media
    while also recognising that it has some faults. Touching on social media within higher education, Akram et al
    discuss that social media allows individuals to share thoughts with others on the other side of the planet instantly
    is a massive positive, and in many cases, this shared information then becomes easily available for many others to
    see and benefit from. The literature saw that social media helped in development towards simply being more prepared, stating that
    social media is fundamentally about showcasing and taking part in  current trends around the world, further enabling students
    to plan or gauge an idea of what might be expected of them. In contrast to those points, the paper outlines that
    social media could aid in reduced learning and research capabilities.


    With a growing dependency on information being easy to find, this could hinder the development of research skills.
    In most cases, people tend to use slang or abbreviated language on social media as most relationships between individuals tend to be interpersonal, coupled
    with an increased reliance on spellcheckers and auto-correction, which decreases their charge over the dialect and
    formal writing abilities. Another valuable note from the paper shines a light on time wastage, while social media
    and the internet, in general, are a boon for education, it opens the door to many distractions if the right amount
   of self-discipline is not present. While social networking has improved the quality and rate of coordinated
    efforts from students, it remains important to be responsible and understand the possible negative effects
    that social media brings forward. A consensus is established with Tariq et al's \cite{Tariq et al 2012}
    points on the factor of distraction before concluding that people are advised to adopt the positive aspects of
    social media while avoiding the negatives, to benefit from the latest and emerging technologies.

    Kaur and Bashir \cite{Bashir et al 2015} reiterate that there are many benefits to social media use
    for adolescents and that there are also multiple potential risks that can act in detriment to their
    mental health and well-being. Some of the positive impacts outlined were access to health information,
    enhanced communication and learning opportunities, while the negatives involve stress, depression,
    suppression of emotional awareness, fatigue, online harassment, a decline in intellectual ability
    and a shrinking capacity for concentration. The paper goes on to express that these risks could be
    navigated successfully with proper handling, such would call for education around social media abuse
    being provided (expressly to parents to mitigate the risks for children and adolescents), but a greater
    awareness of the capacity for potential harmful effects that these platforms can inflict must be attained.

    Bashir and Bhat \cite{Bashir et al 2017} touch on these points again exploring the psychological effects
    of social media, asserting that excessive use of social media can lead people to disastrous results, starting
    with anxiety and leading to depression. They saw in Pantic et al.'s study \cite{Pantic et al 2012}, mirrored
    by a study conducted by Rosen et al. \cite{Rosen et al 2013}, that depression and time spent on Facebook by
    adolescents were directly correlated and those that spent most of their time engaging in online activities
    were among those with major depressive symptoms. The paper concludes to suggest that social media can have an
    epidemic-like on any individual. Social networking sites should be constrained to an age limit and any
    social media application that does not have a positive impact and enables such things as discrimination, violence
    and racism should be dissolved at once, completely.

    Naslund et al. \cite{Naslund et al 2020} found in their paper that in young people, many benefits of social
    media were recognised. Among these benefits were elevated self-esteem and more opportunities for
    self-disclosure. Some of the negative aspects observed were increased exposure to harm, depressive symptoms,
    social isolation and bullying. The paper concludes by highlighting that social media has become an important part
    of the lives of many individuals living with mental disorders and that many of them use social media to share
    lived experiences regarding mental illness, seek support from others, find treatments and recommendations as
    well as access to mental health services. It also recognises that social media platforms could be used to
    allow individuals to access evidence-based treatments and support. The authors end on the recognition that to
    fully close the gap in mental health services integrated into social media, it would require researchers
    to work closely with clinicians to make sure that the benefits of such things on social media platforms would
    outweigh any possible risks.


\section{Evaluation}

        \subsection{Social Media in higher Education}
        Almost all papers found positive outcomes with integrating social media into higher education and in most cases, the utility and ease
        of collaboration are at the centre of praise. It is recognised that each platform has something different to offer and we are perhaps
        better off seeing them as tools each in their own regard, built for their own purpose. With that said, one characteristic shared among these platforms
         is the opportunity to easily reach others from across the globe whilst still making effective team management and organization possible and instant.
         The studies observed noted an increase in team ethic with the introduction of social media in their courses and a direct correlation between
         social media usage and course engagement while their online activities showed no evidence of being related to interpersonal relationships
         between students and their tutors.

	 Another consensus is that to fully leverage social media within higher education, a balance must be found.
         Social media sites play host to many distractions and irrelevant activities and nonsensical information, calling for the responsibility of domain
         selection, and responsibility for proper use. In the same vein as striking balance, setting and formality must be taken into consideration, different
         approaches and rule sets may need to be applied between settings of different formalities and would not be a one size fits all scenario. To fully
         employ social media in pedagogy, the staff and faculty would be at the heart of operations, it would require a small degree of technical competence and
         familiarity for teaching staff to use such platforms effectively and engagingly.

         All literature reviewed conducts a study on existing mediums of social media where distractions already exist and the courses may have altered slightly  to
         fit into social media space. For this research project, I will build a platform expressly for use within higher education,
         and conduct my study in the same fashion as those previously mentioned.

        \subsection{Effects of social media}
        All studies reviewed seem to agree that there is a need for a greater understanding of the risks of social media to fully
         be able to leverage its many benefits. Social media sites if left unmonitored can be the prime environment for negative things to thrive
         such as bullying and discrimination. There are also the less directly harmful aspects to consider, such as excessive amounts of advertisements that
         cause distraction and promote time-wasting and other non-productive activities. It is engaging in these activities and succumbing to the lure of
         these advertisements that gives social media its addictive nature, which if left unchecked can lead to conditions such as depression and anxiety.
         To satisfy the concerns outlined by the literature, a platform would need to be created with these points at the forefront of its design.
         It would need to be engaging but not distracting, a place safe from harm and free from discrimination and bullying, and host activities on
         which time spent is time well spent.





\section{Artefact Design}

        The title of the platform is ``myUniSocial", the core of which the system is an evolution of my work carried out in developing a previous
        social media system \cite{Daley 2022}.The previous system or `myCircle'\cite{myCircle} is built around interests and hobbies as opposed
        to simple media reposting to navigate the negative impacts observed in the above review of past literature and the comparison culture that exists in modern
        society. In recognition of the distraction factor mentioned by many in the literature review, it felt like somewhat of a responsibility
        to minimise possible distractions on the platform, and if they do exist, to make them a benefit to engage with and for it to be relevant
        to their course. To achieve this, in place of the interest groups or `circles', course modules are displayed allowing users to
        quickly access course content and helpful materials. Work from my other  platform `myUni404' \cite{myUni404} is built in as a feature to aid students
        in various fields of computing, this platform offers users the ability to post coding, programming and tech related questions which users can in turn reply to answer with code snippets to
	supplement their answers, and is browsable by speciality ie. -  Web Development, Game Development, Robotics and so on. `myUni404' is built with a beginner-friendly focus and this will be
	a great step in the direction of making the social network educational while building confidence in expertise.
	The layout of the user interface of the platform is designed to have a familiar feeling experience to what users have generally come to expect from social media platforms, with myUni404 being an 
	improved version of the myCircle user interface, the myCircle elements were updated to include the improvements to achieve a consistent feel of quality across the platform. 
	Colour schemes that somewhat align with the university were used to give users a sense of what the platform could feel like if the platform was backed by the university and was the new
	standard for a student portal-based platform. Initially, university logos were on the platform login and navigation bar, but were removed after taking trademarks and other such things into
	consideration.

        \begin{figure}[h!]
                \includegraphics[width=\linewidth]{myCircle.PNG}
                \caption{myCircle user interface}
                \label{figure 1}
        \end{figure}
	
	 \begin{figure}[h!]
                \includegraphics[width=\linewidth]{images/myunisocial.PNG}
                \caption{myUniSocial user interface}
                \label{figure 2}
        \end{figure}


        In appendix D figure 11 a class diagram shows the base system, with the main functionality coming from 'App.js' and being passed into
        functional child components. This class diagram is a derivative of the previously mentioned 'myCircle' system with the relevant adjustments before the merging of the myUni404 platform,
        as is the same for the Use Case Diagram which can be seen in Appendix E figure 13. The growth of the system after introducing the second platform can be observed by comparing Appendix D figure 9
	 with Appendix D figure 12.

        \subsection{Technological Stack}
        The stack used for the platform is a React.js front-end \cite{React} making use of Material UI for styling \cite{Mui}, alongside a Node powered back-end \cite{Node} with Express for server
    functionality\cite{Express}. The project utilises a mySQL database for its capabilities of working with data relationally. Other technologies are at play as dependencies such as Axios for front end to
     back end requests, a full list of technologies can be found on the project github link in Appendix H.





\section{Artefect Development}
	As this project was an evolution of the merging of two projects (myCircle and myUni404) it took around 12 weeks to finish development to a satisfactory level. The front-end or user interface of
	the platform was built in React using JSX, making use of Material UI for some styling, while the back-end and server-side code was written in Javascript with endpoints being made using
	Express. The development cycle was carried out using an Agile approach utilising weekly sprints to add and refine functionality. The overall layout of the user interface for myUni404 was based 
	on that from myCircle with some	improvements. A CSS grid was used to allow for on-screen elements to scale proportionally regardless of screen size or resolution. Using the improved and more
	evolved form of the layout across the entirety of the platform was a necessity as it looked a great deal better and increased functionality, the refactoring of the myCircle components took
	precedence at the start of development.	Following suit with the user interface, the back-end and server-side codebases of the two platforms were combined. myUni404's server code was built on
	top of its predecessor, so the vast majority of the front-end components were built to plug directly into back-end code base with the exception of some renaming of variables or endpoints. The
	original myCircle server code was one file consisting of around 1200 lines which grew considerably with the introduction of the code from the second platform. Having learned better practices and
	striving to meet and maintain industry standards, a large refactor was essential, endpoints were grouped by area of responsibility (ie relating to account settings, fetching or working with feed
	data etc.) and split into separate modular files. This resulted in a massive increase in codebase maintainability and making the project much easier to work with. Further improvements to the
	back end of the site refining how browser sessions were implemented. Browser sessions allow the web browser to store small pieces of data about the platform so that the website can remember
	things like the users' username, profile picture and whether or not they have an active session, removing their need to log in after every page refresh and in turn improving the overall user
	experience of the platform. The previous version of the chat system relied on the page making a request to the back end and updating regardless	of there being any new messages or updates
	waiting. Granted this system worked but in its nature added an unnecessary load on the server. In its place, the new system utilised web sockets to allow for real-time updates with no extra
	requests to the server as there is a single pipeline of communication established between the two users.

	With the system being available to users throughout the experimental phase, system security was of the highest priority. As such the server and hosting solution was altered to use HTTPS protocol
	 to add encryption to HTTP requests and responses, vastly increasing end-user security while using the platform. 
	 A `Sign out \& delete data' button was also added for users to be able to wipe all of their data from the server on completion of their participation.`
       
       \subsection{Testing of artefact performance}
      Artefact front-end components were subject to unit tests which can be seen in Appendix B figure 8. The most frequently used and constant of components were tested and all testing suites produced
      passing outcomes. Google Lighthouse was used to test overall page performance and accessibility for the feed page, the output of this test can be seen at Appendix B figure 9.

      The login form was also stress tested with 10,000 requests at a rate of 200 requests per second. Login information for a test user was passed into the test to see how the login form and 
      database handled the load of such a volume of requests. Given that hosting solutions were limited and the server played host to important files and other projects, it was not an option at
      this time to stress test the deployment on the production server, this test was performed on a local deployment. The command could be performed in the same fashion seen in
      Appendix B figure 10 on a staging environment, the figure shown displays the output given for the local instance.

        \subsection{Validation and Verification}

        To ensure the validation and verification of the proposed system, the application will be built in line with ISO/IEC 25010 \cite{ISO25000} standards.
        These standards ensure that elements such as functionality, performance, security, maintainability etc. are satisfactory to ensure the development of a high quality system.

        The product quality model defined in ISO/IEC 25010 consists of the quality categories seen in figure 2, below.

        \begin{figure}[h!]
                \includegraphics[width=\linewidth]{images/iso.png}
                \caption{Quality Characteristics of ISO 25010 - https://iso25000.com/images/figures/en/iso25010.png}
                \label{figure 2}
        \end{figure}





\section{Research Methodology}

	\subsection{Research Question}
		Could an institutionalised social media platform benefit the academic and social aspects of university life?

	\subsection{Hypotheses}
		\subsubsection{Hypothesis}
		The prototype platform has a positive effect on the social and academic aspects of university life.
\\
		\subsubsection{Null Hypothesis}
		The platform has no effect, or a negatice effect on the social and academic aspects of university life.


        \subsection{Philosophical Position}
        Having great applications within sociology and psychology, the philosophical approach of this study was taken from an 
        interpretivist point as the application of this platform and the observations of its utility somewhat align with the social
         sciences and as such will require a socio-scientific approach. This interpretivist approach by its nature called for a qualitative
         method; often criticised for its generalisability and flaws in reliability, quantitative and qualitative methods were combined
         in order to increase the reliability and validity of findings ultimately making this a mixed-method study.

        \subsection{Experimental Design}
        To gather information on the effectiveness of my platform, I conducted a between-groups study involving
	two study groups. An ideal sample size of 52 participants (2 groups of 26) was recommended by G*Power to establish statistical significance with a confidence interval of 95\%
	for this study. See figure 4 for a depiction of a central and noncentral distribution plot, see Appendix F figure 18 for full G*Power output.
	Both groups were given the same set of tasks to complete, group A were asked to fulfil
        these tasks on the existing online university portal ie. Learning Space or Moodle while group B were asked to 
        do so on the prototype platform. An online survey was used to collect information on the test users' experiences
        of both platforms in an effort to gauge how much of a benefit they felt the systems offered in both the
        learning and social aspects of university life. The survey was conducted in a similar manner to
        studies explored in the review of literature \cite{Liu2010}\cite{Baruah2012}\cite{Wang2011}\cite{Evans2014}\cite{Akram et al 2017},
        the results obtained from those papers seemed proficient and I felt it well justified the survey as a means for collecting data.
        A between-subjects study was chosen as each group was testing only one platform as a cohort and were asked to test all functionalities of
        the system as per the tasks. Users were asked to complete a Likert Scale style survey and a qualitative survey to fully convey their experience.
	        \begin{figure}[h!]
                \includegraphics[width=\linewidth]{images/gpowerplot.png}
                \caption{Central and noncentral distributions G*Power plot}
                \label{figure 3}
        \end{figure}
	\subsection{Legal, Social \& Ethical considerations}

	\subsubsection{Legal}
        Database entries for users, required for functionality of the artefact were handled responsibly. Users' passwords were hashed and salted to ensure that in
	the event of a database breach,	the entries that the attacker would see could not resemble the actual data, and a server-side pepper was also at play to
	prevent such breaches from occurring. All regulations defined by GOV.UK and the Data Protection Act 2018 \cite{Gov UK Data Protection} were met and abided by throughout
        the development and life of the platform. More sensitive information such as race, ethnic background, religious beliefs and other more
        sensitive categories mentioned on the cited website were not stored for functionality of the platform as such information
        is irrelevant for the platform to function properly. A full privacy policy and cookie policy was available to users via the footer on the login screen or
	through the account settings menu, informing users that their data was to solely be used for functionality within the platform, such as displaying
	their name, email for log in authentication and so on.

        ICO (Information Commissioners Office) \cite{ICO} Data protection self-assessment for Data Controllers, Data Processors and Information Security
        checklist reports gave an overall rating of green, with all required Data Control and Security measures set to be in place and that
        there are no other parties receiving, processing or controlling user data from my platform. Any data held as part of research conducted was held to such standards,
	regulations and was considered an utmost priority.
\\
        \subsubsection{Social}
                All measures were taken to ensure that opportunities for unwanted or harmful behaviours to take place (ie. bullying and discrimination)
                were minimised during testing and data collection to ensure that the platform was a safe space for its users.
\\
        \subsubsection{Ethical}

		\paragraph{Informed Consent \& Voluntary Participation}
		Participants were required to complete a consent form before commencement of their participation. This consent form outlined the general purpose
		of the study, the purpose of the project and how the findings will be used while also ensuring their understanding that their participation is
		voluntary and that they maintain the freedom to withdraw their participation at any point and for any reason without explanation.
     
      		 \paragraph{Do no harm}
       		The evaluation process was carried out in such a manner to minimise any potential causes of harm (unintentional or otherwise) to participants.
		These potential harms could have taken form by subjecting users to undue stress, pain, anxiety, diminishing of self-esteem or invasions of privacy.
		These risks were mitigated by ensuring that no required tasks or questions required as per the data collection survey were invasive, controversial
		or damaging in nature and were generally everyday tasks that students and users would typically perform as part of their student life.

		\paragraph{Confidentiality}
		Any identifying information of participants, within my research data and the prototype platform, have not been made available to or accessed
		by anybody apart from myself. As such all documentation and reports exclude any identifying user information in due respect to both participant
		and user confidentiality.

       		\paragraph{Anonymity}

       		Participants' data submissions were anonymous and no identifying factors were necessary or relevant as part of the questions or tasks that users 
		were asked to perform.

		\paragraph{Only assess relevant components}
		Only relevant components of functionality were assessed as part of this study, as such users were only asked to complete tasks which were able to be
		accomplished on both platforms being tested to avoid any biases that could come about by tailoring the questions in favour of one platform over 
		another.
	\subsection{Data management}
		Data for this study was collected by means of an online survey which consisted of both qualitative and quantitative questions. An online survey was chosen for this study to follow suit
		with studies conducted in previous research within this area  \cite{Liu2010}\cite{Baruah2012}\cite{Wang2011}\cite{Evans2014}\cite{Akram et al 2017}. The quantitative
		questions were in the form of a Likert scale, asking users to rate their experience of performing a certain task on a scale of 1 (least favourable/difficult)
		to 7 (most favourable/easy), while the qualitative questions asked users to explain a little about their experience of performing
		 that task.

		Submissions were collected by using Microsoft forms and kept securely in a Microsoft Onedrive. At the end of the experimental phase any remaining
		identifying factors were removed from the dataset to ensure the security of participant data while working with any results on my local machine.

\section{Results}

 The survey to collect data consisted of both qualitative and quantitative questions. These questions were generally coupled with a quantitative question first, asking them to perform a task and then 
 rate their experience of doing so on a Likert scale, followed by a qualitative question asking them to explain a little about their experience of performing the task. Two exceptions to this are questions
 6 and 7 which ask how likely the user would be to engage with the community and how likely they would be to engage with course-related content respectively.
 \\
 \\
 The quantitative questions are as follows:
\\
\centerline{
    \begin{tabular}{|c|c|}
            \hline
      	    \thead{Question No.} & \thead{Question Text} \\
            \hline
	    Q1 &  \makecell{Rate the difficulty of finding course\\ related content on the platform.} \\
      	    \hline
	    Q2 & \makecell{Rate the difficulty of finding a forum/\\conversation about your modules on\\ the platform.} \\
	    \hline
	    Q3 & \makecell{Rate the difficulty of accessing your\\ account settings on the platform.} \\
	    \hline
	    Q4 & \makecell{Rate how much you feel the platform might\\ benefit your education.} \\
	    \hline
	    Q5 & \makecell{Rate how much you feel the platform might\\ benefit your social life at university.} \\
	    \hline
	    Q6 & \makecell{How likely are you to engage with the\\ community on this platform?} \\
	    \hline
	    Q7 & \makecell{How likely are you to engage with course\\ related material on this platform?}  \\ 
	    \hline
    \end{tabular}
}
	\subsection{Data Analyses}
		With the Likert scale ranging from 1 (least favourable answer) to 7 (most favourable answer) users generally had a higher overall satisfaction rate on the prototype platform `myUniSocial'
		with the exception of the final question, ``How likely are you to engage with course-related material on this platform", which came in at a very close second. Figure 4 displays the mean scores
		of all quantitative questions with group A representing the existing student portal and B depicting the scores from the prototype platform.

		\begin{figure}[h!]
			\centerline{
				\includegraphics[width=\linewidth]{images/GroupedBarComparison.png}}
        		\caption{All quantitative question scores across both platform. A = Student Portal, B = Prototype Platform}
			\label{figure 4}
		\end{figure}

		\begin{figure}[h!]
        		\includegraphics[width=\linewidth]{images/PrototypeFull.png}
        		\caption{All quantitative question scores for prototype platform}
        		\label{figure 5}
		\end{figure}

		\begin{figure}[h!]
        		\includegraphics[width=\linewidth]{images/StudentPortalFull.png}
        		\caption{All quantitative question scores for the student portal}
        		\label{figure 6}
		\end{figure}

		Comparing figures 6 \& 7 shows that there is generally a widespread within the results, but the majority of the more favourable results are still claimed by the prototype platform.
		This is especially prevalent in any questions related more toward the social aspects of university, most notably questions 2, 5 and 6. The student portal scored higher on question 7
		which asked users how likely they were to engage with course-related content on the platform.


\section{Discussion}

With the prototype platform generally having a higher overall score across all questions, it is suggested with some significance that an institutional social media platform centred around the university
could indeed be of a benefit to both social and academic aspects of university life. Participants reported that even having used the existing student portal for some years, they were either unaware of the
forum functionality that was in place or that it was simply not enticing to use without it being a criterion for sessioned work. The consensus was that participants saw very little to no benefit
to their social lives within the student portal.

Some of the lesser scores for the prototype platform were generally more centred around finding course-related content or using the platform as a general aid for their studies as they would the
student portal, and were typically coupled with qualitative responses explaining that they could not find course content specifically relating to them or their chosen track of study and modules.
For this study to gain more concrete results there would be a requirement for extra clearance and permissions to access university-wide module and student lists or information in order to
populate the modules with the correct students, in turn being on equal footing with the student portal and not ultimately comparing a literal against a hypothetical.

The prototype platform witnessed a good level of engagement with some users posting questions into the myUni404 space, changing account profile pictures, posting and generally engaging with other
students. This level of interaction could further support Kelm's \cite{Kelm2011} study in which social media was incorporated into course material and noted an increase in course engagement and sense of
team ethic. Participants seemed to have no issues finding their way around the prototype platform or with any aspects of the usability of the system, evidence for this is suggested in the quantitative
results as users appear to have experienced more difficulty in navigating certain features of the platform within the student portal than on the prototype social media platform.

Qualitative results from the prototype platform also mention that having to sign up and register as a user, in contrast to students already having an account pre-registered on the student portal
at their arrival to university, left participants feeling that it would be another platform to manage in addition to the social media sites that they regularly use during their everyday lives.

Improvements could be made to pull more robust outcomes should the study be iterated over again in an effort to remove any potential unconscious biases. Improvements that I would make would include; making
Likert scale range from -4 to 4, or perhaps removing the numbers from the scale entirely in favour of a slider that starts at a neutral position and implementing some Human Computer Interaction
(HCI) analytics software to generate heat maps regarding mouse movements while using the platform to gauge how easily participants were able to navigate the platform.

Given the small window of time to build the artefact and conduct this study, this research could be regarded as a viable precursor to a seemingly promising area of research. With more time, mitigation of 
limitations such as; having support and access to resources from the institution regarding permissions to effectively redistribute course content, student records to facilitate automatic profile and
account generation as well as populating modules on the platform with the correctly enrolled students, we could definitively answer the research question: Could an institutional social media platform be of a benefit
to the social and academic aspects of university life.


%\begin{itemize}
%        \item Reiterate the research problem and scope
%        \item State and synthesize the major findings.
%        \item Explain the meaning of the findings.
%        \item Relate the findings to prior art and studies.
%        \item Defend the interpretation, considering alternatives.
%        \item Articulate the broader implications for research, theory, and/or practice.
%        \item Suggest directions for future research.
%\end{itemize}


\section{Future Work}
	If this study were to be continued and explored further, adjustments should be made to improve the prospects of acquiring more meaningful results. Modifying the survey Likert scale to range 
	from -3 to 3 or removing numbers from it entirely in place of a slider with a neutral origin could help remove any unconscious biases.

	Making the research question more targeted towards a single and more definitive answer, ``benefiting social and academic aspects of university" is perhaps too broad for one study,
	or this may in fact be grounds for A B style testing to measure	those metrics separately.

	Limitations and restrictions would need to be mitigated for this study to continue any further, access to module and enrolled students or building a system that simply
	`plugs in' would serve this study well, as students not being able to find their course or module ultimately hampered my results. University logos on the platform  would help 
	users feel more immersed in an online university environment and accounts being pre-registered could in turn pacify users that reported feeling like it would just be another social media
	platform to maintain. Permission to use and essentially redistribute course materiel, in lieu of a system that `plugs in', would vastly improve the quality of this study as in doing so would
	remove the current need to compare a literal (the university student portal) with a hypothetical (the prototype platform).

	Even with such limitations, restrictions and areas for improvement, this remains an interesting area of study and users took the prototype platform well. Even with the platform being a
	hypothetical, students embraced what it could be and engaged with the available content, posted questions and collaborated with one another on a system that they felt to be intuitive and
	somewhat natural to use. This is evident in both the qualitative and quantitative data, but it would be advantageous to run some Human-computer interaction (HCI) analytics to measure
	this more candidly.


\section{Conclusion}
	Acknowledging that this study compared a literal university student portal with a hypothetical prototype student portal, it cannot be concluded with certainty that an institutional
	social media platform could benefit both social and academic aspects of university life. Nonetheless, I feel that given the results, there is evidence to suggest that if studied further 
	with the previously mentioned limitations and restrictions mitigated that it would be a welcome evolution of systems currently being used and in turn see an increase in student/course engagement
	while promoting student collaboration.
	The prototype platform saw good levels of engagement with users updating their profile information, changing profile and cover pictures, posting and making use of the `myUni404' section of the
	platform which allows users to post programming or tech-related questions that include a code snippet as well as users also replying to such content. This behaviour is promising in terms of the 
	potential for collaboration that such a platform could provide and seemingly calls for a redesign or reformat of how student portals function for students at university.

\section{Acknowledgements}
I would like to thank Sokol Murturi, Dr Joseph Walton-
Rivers, Dr Michael Scott and Warwick New as well as all
other staff of the Games Academy at Falmouth University for
making this project possible and guiding me in bringing it into
fruition. 


\begin{thebibliography}{}
\bibliographystyle{IEEEtran}
\bibitem{Discord}
        Discord. {\it{https://discord.com/}}
\bibitem{Liu2010}
        Liu, Youmei {\it{Social Media Tools as a Learning Resource}} Journal of Educational Technology Development and Exchange (JETDE): Vol. 3 : Iss. 1, Article 8, 2010

\bibitem{Baruah2012}
    Baruah, Trisha Dowerah {\it{Effectiveness of Social Media As a Tool Of Communication And Its Potential For Technology Enables Connections: A Study.}}, New York, NY, USA: Springer, 2007.

\bibitem{Kelm2011}
    Kelm, Orland R. {\it{Social Media: It's What Students Do.}}Business Communication Quarterly, Volume 74, Number 4, December 2011.

\bibitem{Wang2011}
        Wang, Qingya / Chen, Wei / Liang, Yu {\it{The Effects of Social Media on College Students}}.MBA Student Scholarship, Paper 5. 2011.

\bibitem{Evans2014}
        Evans, C. {\it{Twitter for Teaching: Can Social Media Be Used To Enhance The Process Of Learning?}} British Journal of Education Technology Vol 45, No 5, 2014.

\bibitem{Williams2022}
        Williams, R. Thomas {\it{Social Networking Services (SNS) In Education}}
        Asian Journal of advances in Research. 17(1): 1-4, 2022.

\bibitem{Junco et al 2013}
        Junco, R. / Elavsky, C. Michael / Heiberger, G. {\it{Putting Twitter to
        the test: Assessing outcomes for student collaboration, engagement and
        success.}} British Journal of Education Technology. Vol 44 No 2 2013

 \bibitem{Tripathi 2022}
        Tripathi, Dr. Sheel Nidhi {\it{Social Media in Higher Education}}
        Communication Today, January-June, 2022.

\bibitem{Haythornthwaite et al 2016}
    Haythornthwaite, C. / Paulin, D. / Gruzd, A. {\it{Analyzing Social Media
        and Learning Through Content and Social Network Analysis: A Faceted
        Methodological Approach.}} Journal of Learning Analytics, 3(3), 47-71.
        2016.
\bibitem{Amedie 2015}
        Amedie, J. {\it{The Impact of Social Media on Society.}} Advanced Writing: Pop Culture Intersections. 2 2015.

\bibitem{Kuppuswamy et al 2010}
        Kuppuswamy, S /Shankar Narayan, P. B. {\it{The Impact of Social Networking Websites on the Education of Youth.}}
        International Journal of Virtual Communities and Social Networking, 2(1), 67-79.
        2010.
\bibitem{Tariq et al 2012}
        Tariq, W. / Mehboob, M. / Khan, M. Asfandyar / Ullah, F. {\it{The Impact of Social Media and Social Networks on Education and Students of Pakistan.}} International Journal of Computer Science Issues, Vol. 9, Issue 4, No 3.
        2012.
\bibitem{Ellison et al 2007}
        Ellison, N. / Steinfield, C. / Lampe, C. {\it{The Benefits of Facebook "friends:" Social Capital and College Students' Use of Online Social Network Sites.}} Journal of Computer-Mediated Communication, Vol 12, 4.
        2007.
\bibitem{Pempek et al 2009}
        Pempek, Tiffany A / Yermolayeva, Yevdokiya A. / Calvert, Sandra L. {\it{College students’ social networking experiences on
                Facebook.}} Journal of Applied Developmental Psychology, Vol. 30, Issue 3, page 227–238.
        2009
\bibitem{Akram et al 2017}
        Akram, A. / Kumar, R. {\it{A Study on Positive and Negative Effects of Social Media on Society.}}
        International Journal of Computer Sciences and Engineering, Vol 5, 10.
        2017
\bibitem{Bashir et al 2015}
        Kaur, R. / Bashir, H. {\it{Impact of Social Media on Mental Health of Adolescents.}}
        International Journal of Education, 5, 22-29.
        2015
\bibitem{Bashir et al 2017}
        Bashir, H. / Bhat, S. A. {\it{Effects of Social Media on Mental Health: A Review}}
        The International Journal of Indian Psychology, Vol 4, Issue 3.
        2017
\bibitem{Pantic et al 2012}
        Pantic, I. / Damjanovic, A. / Todorovic, J. / Topalovic, D. / Bojovic-Jovic, D. / Ristic, S. Ristic/ Pantic, S.
        {\it{Association Between Online Social Networking and Depression in High School Students: Behavioural Physiology Viewpoint.}}
        Psychiatria Danubina, 24(1), 90-93.
        2012.
\bibitem{Rosen et al 2013}
        Rosen, L.D. / Whaling, K. / Rab, S. / Carrier, L.M. / Cheever, N.A.{\it{Is Facebook Creating "iDisorders"? The Link Between Clinical Symptoms of Psychiatric Disorders and Technology Use, Attitudes and Anxiety.}} Computers in Human Behaviour, 29, 1243-1254.
        2013.
\bibitem{Naslund et al 2020}
        Naslund, John A. / Bondre A. / Torous J. / Aschbrenner, Kelly A. {\it{Social Media and Mental Health: Benefits, Risks, and Opportunities for Research and Practice.}} Journal of Technology in Behavioral Science 5:245-257.
        2020.
\bibitem{Daley 2022}
        Daley, D. {\it{A Creative Approach to Social Media}}
\bibitem{myCircle}
        myCircle
        {\it{http://dd252935.kemeneth.net:9010/}}
\bibitem{myUni404}
    myUni404
    {\it{http://147.182.247.158:9010/}}
\bibitem{React}
        React JS
        {\it{https://reactjs.org/}}
\bibitem{Mui}
        Material UI
        {\it{https://mui.com}}
\bibitem{Node}
        Node JS
        {\it{https://nodejs.org}}
\bibitem{Express}
        Express JS
        {\it{https://expressjs.com}}
\bibitem{Agile}
        Agile Development Method
        {\it{https://www.atlassian.com/agile}}
\bibitem{Jest}
        Jest JS
        {\it{https://jestjs.io/}}
\bibitem{Mocha}
        Mocha JS
        {\it{https://mochajs.org/}}
\bibitem{SuperTest}
        SuperTest
        {\it{https://www.npmjs.com/package/supertest}}
\bibitem{ISO25000}
        ISO 25000 Software and data quality.
        {\it{https://iso25000.com/index.php/en/iso-25000-standards/iso-25010}}
\bibitem{Gov UK Data Protection}
        Gov.UK. {\it{https://www.gov.uk/data-protection}}
\bibitem{ICO}
        Information Commisioner's Office. {\it{https://ico.org.uk/for-organisations/sme-web-hub/checklists/data-protection-self-assessment/}}
\end{thebibliography}
\appendices
\newpage
\clearpage
\newpage

\section{Reflective Report}
 
This project ``myUniSocial" is an evolution of my previous project ``myCircle". Since starting my journey to become a web developer this type of project was always an
area of interest for me as I feel social media systems, as we have come to know them, incorporate a vast range of elements and processes that I see to be at the core of many
web applications.
Overcoming challenges such as; how to store a users “friends list” without storing more than one name in a database field, 
Implementing real-time chat systems, 
Ensuring database data is kept securely, browser sessions, implementing HTTPS protocol for security and 
Restricting content based on a particular users' friend list were all part of bringing this project to life. 
In overcoming these challenges and acknowledging the broader applications of these elements and challenges, I can say that I chose the right project to pursue and feel that it has made me a better developer.

\subsection{Cognitive}
Having attempted to build a social media platform for my final second-year personal project, I approached the development of this platform with the knowledge of what parts to build. The
first version worked well for the most part, but parts of the system had much room for improvement. Parts of the user interface were unsatisfactory, the layout was inconsistent across screen
resolutions, the server code was all in one file and the chat system worked by sending a request to the back end every 5 seconds or so regardless of any updates that may or not be waiting.
Acknowledging these issues along with the required components and moving parts, I knew exactly what the new system needed to feel more complete and boast a higher quality, and how to go about it.
I understood that the front-end system needed to be built in a more modular way to get a satisfactory user interface across all screen resolutions. Working with the previous server code-base
taught me to make modular server files from the get-go to acheive an easier and more maintainable development environment. Web sockets in place of the old chat system allowed for real-time
communication while also reducing load on the server. All of these valuable updates came from the knowledge and understanding gained from the first iteration of building a social media system.
An appropriate smart goal to further improve in this area could be to take more time to whiteboard and wireframe system architecture or components to better plan how the system or components should 
be made for the group project currently in development.
\\Specific: I will take time to whiteboard and plan new components for the group project to increase quality of system architecture.
\\Measurable: This could be measured by observing a tidier codebase, cleaner solutions and a better system architecture.
\\Achievable: Seeing how the implementation of a modular approach effected this project, applying that more rigorously to this project is very acheivable
\\Relevant: Being able to think and plan around system architecture is very important to be a successful and competent web developer.
\\Time-bound: The group project has a deadline for a little over a month away, giving this goal a concrete deadline.

\subsection{Procedural}
The “friends list” problem was one of the  challenges to overcome within developing the platform. Being poor form to store more than one item of data in any given field in an SQL database,
figuring out how to store or gather a list of users' friends required a lot of thought. Whiteboarding the problem, I ultimately had the idea of storing `friendships` in their own separate
table. These friendships weren’t attached to the users' table but stood aside from the users  consisting of two fields, user1 and user2. With this in place, to gather a user's friends
list ‘Get all entries from the friendships database where user1 or user2 matches the username of the logged in user’. With that data gathered ‘remove all names from that returned list
that matches the logged in user's username, leaving only the names of the users names. This way around the problem felt particularly clean as it works across the board and for all users.
Storing multiple names in a friend list field instead would create a massively bloated database, taking into consideration people in the same friend circles that would have a similar
friends list and essentially be a lot of duplicate data. The implementation I went with ensured that no data is repeated, is minimised ultimately being a clean solution to what could
have been a messy problem. Overcoming this challenge made me recognise a great deal of self-development and vastly increased my confidence in problem-solving which I see to be vital as an 
aspiring web developer.
\\ A smart goal for this area could be to read more documentation. Reading more documentation could reduce the need to think around problems like this, as there is more than likely tried and true 
industry standard ways of dealing with them.
\\Specific: I will read documentation for the technologies and libraries used as part of the group project.
\\Measurable: With a list in the project of what technologies are currently at play within the group project, this goal can be measured accurately against it.
\\Achievable: This goal is very achievable as for the most part the documentations are easily accessed, and I have a list in the repository of what has been used.
\\Relevant: As a web developer, a good understanding of technologies and frameworks is incredibly important making this goal very relevant to my career.
\\Time-bound: The deadline for the group project is a little over a month away, giving this goal a set deadline to be completed in.
\subsection{Affective}
Developing the platform was something that I genuinely enjoyed doing. Seeing the platform grow and the components working as intended felt like achievement after achievement. That said, the want to keep 
building and get things `just right' may have got in the way of other areas of this project. Wanting research participants to use and see the platform as envisioned ultimately delayed the 
research part of this study and forced hasty decisions to be made such as quantitative survey questions that would have benefited from more time and thought. In the end my pride and perfectionist 
approach to developing the artefact reduced the significance of the overall study whereas having users fulfil the research phase on a minimum viable product would have produced better results.
The scope for this project was definitely ambitious and given the opportunity to do it all again I would still choose a project of this nature, but I would manage myself slightly more strictly in terms 
of getting carried away on the artefact and pay more attention to the other elements of the study.\\ A smart goal for this could be to more strictly focus on producing an MVP for the group project right
away to avoid other areas of the project being neglected.
\\Specific: Focus more on producing an MVP.
\\Measurable: This could be measured by having a working platform with reduced functionality sooner, rather than refined components being ready immediately.
\\Achievable: This is very achievable and only requires a shift in focus.
\\Relevant: This focus is very relevant to web development as it would increase client satisfaction by seeing working systems sooner.
\\Time-bound: The group project is due in just over a month, making this goal comfortably time sensitive.


\subsection{Dispositional}
Lack of engagement certainly was not an issue with this research and development project. As previously mentioned working on this project was something that I genuinely enjoyed and I was proud to be 
building what I felt to be a giant compared to other projects I have worked on both individually and as part of a team. With that in mind, and again previously mentioned, I may have been carried away with 
the development of the artefact. The vast majority of my time over the past 3 months have been devoted to the platform's development whilst also making some time for other engagements such as work on other 
projects and internships. Building something like a social media platform is a reasonably large task, and I felt that if it was missing any element that we have generally come to expect from a social media
platform, that absence of that element would be all that people see. With that in mind I was very motivated to make sure that the platform had all of the common features that one would expect to see, with 
the exclusion of voice or video chat.
\\ A smart goal to better manage this would be to adhere to sprint goals more closely, getting a lot of work done can be good, but sometimes can get too ahead and cause
more work for the following sprint. This could be applied to the group project.
\\Specific: Following sprint goals and controlling how many hours per day I work.
\\Measurable: This could be measured by ticking off my given tasks according to the current sprint cycle.
\\Achievable: This goal is very acheivable as it ultimately gives me more free time.
\\Relevant: Sticking to sprint and essentially `following the brief' is very important as a web developer and is vital for controlling workflow.
\\Time-bound: With remaining development time for the group project being around a month, there is a strict timeframe to work with on this goal.

\subsection{Interpersonal}
Having built a good rapport with university staff, my supervisor and the available expertise, good communication and engagement was essential in the development of this project. Building a social media 
platform with university elements built into it to try and show potential for its place within an academic setting comes with its own set of risks and potential pitfalls. Being able to communicate with 
staff to discuss ideas and avoid any sort of privacy or legal issues was paramount. At all stages I tried to obtain a well-rounded group of opinions and suggestions from my peers, staff and supervisor 
to help form what the platform ultimately came to be. All communications from both staff and my supervisor were treated as expert opinions, such as a prompt to remove university logo's from the platform
to avoid any trademark issues regardless of the context of the project, were actioned accordingly although somewhat unfortunate as the platform felt more external and removed some university-style
immersion. As such guidance from peers, staff and my supervisor have been an integral part of this research and development project. 
\\As a smart goal, I will improve communication with other team members on the group project which ends next month.
\\Specific: I will speak to the team once a week throughout the rest of development and note any ideas, issues or concerns.
\\Measureable: This can be measured by taking notes, doing so will help solidify any suggestions or concerns.
\\Achievable: I generally do not have an issue speaking up to the group, but a more conscious effort to increase communication cross the team is very acheivable.
\\Relevant: Communication skills are very important as a web devloper as it helps gain a true understanding of projects, while also making me a better team member.
\\Time-bound: With the deadline for team project being next month, there is an explicit time-frame to acheive this goal.

\subsection{Practice}
With my first attempt at developing a social media platform resulting in a codebase with large files and becoming difficult to maintain and work with, I made a conscious effort to take a more modular 
approach to building the server and an object-oriented approach to front end components. This drastically improved the quality of not only the server and front-end components in themselves, but also 
the quality of maintainability and development environment too. Finding problematic code became much easier, in turn making bug fixing an easier process, resulting in a cleaner workflow and greater 
quality in the project as a whole. A well-kept development environment inspired more commenting and clean up in an effort towards keeping the project lean and easy to access components. Taking more 
pride in this space inspired me to follow good practices such uniform variable, function and class naming conventions to uphold code readability as well as ensuring that any useless code was removed.
Lessons learned from previous project made a huge difference in the overall quality of this platform taking form in more modular files, code maintainability and readability.
\\As a smart goal, I will strive to further improve the quality of my work through means of code maintainability and best practices by the end of development of the group project next month.
\\Specific: The goal is to improve quality of work through means of following best coding practices and codebase maintainability.
\\Measurable: The success of this goal can be measured by observing code maintainability on the group project.
\\Achievable: I have started to work within these practices as part of this project, that becoming second nature is very attainable.
\\Relevant: To become a good web developer, developing these good habits is a must for success in the industry.
\\Time-bound: With the deadline for the project being next month, there is a definitive time and date to measure this goal to.

\subsection{Conclusion}
In conducting this experiment and developing the artefact, I can say with certainty that I have learned a lot. The self development has presented itself in contrast between my past few projects and 
has given me a great deal of confidence in the face of problem-solving and approaching issues that I might not immediately know how to solve. In light of self development, I will strive to continue 
down a growing path and complete the SMART goals set out above. Taking time to whiteboard and think about system architecture, reading documentations for used software, focusing more on producing a
minimum viable product, adhering to sprint goals more closely, increasing communication across the team and striving to improve codebase quality are all goals I will achieve in the weeks to come
and will most definitely be a better developer because of it.


\newpage
\onecolumn

\section{Artefact testing addendum}

	\begin{figure}[h!]
		\centering
                \includegraphics[width=0.4\paperwidth]{images/unittests.png}
                \caption{Front end unit tests}
                \label{figure 3}
	\end{figure}
	\begin{figure}[h!]
		\centering
                \includegraphics[width=0.4\paperwidth]{images/lighthouse.png}
                \caption{Google Lighthouse test output}
                \label{figure 3}
	\end{figure}
	\begin{figure}[h!]
		\centering
                \includegraphics[width=0.85\paperwidth]{images/stresstest.png}
		\caption{Login form stress test}
                \label{figure 3}
	\end{figure}
\clearpage
\newpage

\section{Full R code}
	



\begin{figure}[h!]
		\includegraphics[width=0.7\paperwidth]{images/groupedBarRcode.png}
                \caption{Grouped bar plot in R Studio.}
                \label{figure 4}
\end{figure}
\begin{figure}[h!]
                \includegraphics[width=0.7\paperwidth]{images/prototypeRcode.png}
                \caption{Prototype platform Box plot in R studio.}
                \label{figure 4}
\end{figure}
\begin{figure}[h!]
                \includegraphics[width=0.7\paperwidth]{images/learningSpaceRcode.png}
                \caption{Student Portal Box plot in R Studio.}
                \label{figure 4}
\end{figure}

\clearpage
\section{{Prototype Design UML Diagrams}}

	\begin{figure}[h!]
                \includegraphics[width=0.85\paperwidth]{images/myUniSocialClass.png}
                \caption{myUniSocial Class Diagram}
                \label{figure 3}
\end{figure}
\begin{figure}[h!]
                \includegraphics[width=0.85\paperwidth]{images/myUniSocial404Class.png}
                \caption{myUniSocial Use Case Diagram}
                \label{figure 4}
\end{figure}
\begin{figure}[h!]
                \includegraphics[width=0.85\paperwidth]{images/myUniSocialUseCaseDiagram.png}
                \caption{myUniSocial Class Diagram}
                \label{figure 3}
\end{figure}
\clearpage
\newpage

\section{Prototype platform screenshots}

\begin{figure}[h!]
                \includegraphics[width=0.85\paperwidth]{images/myCircle.png}
                \caption{myCircle user interface}
                \label{figure 4}
\end{figure}
\begin{figure}[h!]
                \includegraphics[width=0.85\paperwidth]{images/myunisocial.png}
                \caption{myUniSocial User interface}
                \label{figure 4}
\end{figure}
\begin{figure}[h!]
                \includegraphics[width=0.85\paperwidth]{images/myUni4042.png}
                \caption{myUni404 `Ask Question` page}
                \label{figure 4}
\end{figure}
\begin{figure}[h!]
                \includegraphics[width=0.85\paperwidth]{images/404screenshot.png}
                \caption{myUni404 Question}
                \label{figure 4}
\end{figure}
%\section{Biography Section}
%If you have an EPS/PDF photo (graphicx package needed), extra braces are
% needed around the contents of the optional argument to biography to prevent
% the LaTeX parser from getting confused when it sees the complicated
% $\backslash${\tt{includegraphics}} command within an optional argument. (You can create
% your own custom macro containing the $\backslash${\tt{includegraphics}} command to make things
% simpler here.)

\newpage

\clearpage
\section{Additional materials}

\subsection{Project Github link}
https://github.falmouth.ac.uk/DD252935/myUniSocial

\subsection{Refactoring}
\subsubsection{Server Refactor Link}
https://github.falmouth.ac.uk/DD252935/myUniSocial/commit/a0e6f0efe3ccdce7e3d7efd9ce74bae0ac874be3
\subsubsection{Commit}
a0e6f0efe3ccdce7e3d7efd9ce74bae0ac874be3

\subsection{G*Power Output}
\begin{figure}[h!]
                \includegraphics[width=0.5\paperwidth]{images/gpoweroutput.png}
		\caption{G*Power output}
                \label{figure 4}
\end{figure}

\vspace{11pt}

%\bf{If you include a photo:}\vspace{-33pt}
%\begin{IEEEbiography}[{\includegraphics[width=1in,height=1.25in,clip,keepaspectratio]{fig1}}]{Michael Shell}
%Use $\backslash${\tt{begin\{IEEEbiography\}}} and then for the 1st argument use $\backslash${\tt{includegraphics}} to declare and link the author photo.
%Use the author name as the 3rd argument followed by the biography text.
%\end{IEEEbiography}

\vspace{11pt}

%\bf{If you will not include a photo:}\vspace{-33pt}
%\begin{IEEEbiographynophoto}{John Doe}
%Use $\backslash${\tt{begin\{IEEEbiographynophoto\}}} and the author name as the argument followed by the biography text.
%\end{IEEEbiographynophoto}

\vfill
\end{document}

